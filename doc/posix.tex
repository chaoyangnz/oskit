%
% Copyright (c) 1997-2001 University of Utah and the Flux Group.
% All rights reserved.
% 
% The University of Utah grants you the right to copy and reproduce this
% document or portions thereof for academic, research, evaluation, and
% personal use only, provided that (1) the title page appears prominently,
% and (2) these copyright and permission notices are retained in all copies.
% To arrange for alternate terms, contact the University of Utah at
% csl-dist@cs.utah.edu or +1-801-585-3271.
%
\label{posix-lib}
\label{extended-posix-api}

\section{Introduction}

The \posix{} library adds support for what a \posix{} conformant system
would typically implement as system calls. These \posix{} operations are
mapped to the corresponding \oskit{} COM interfaces. Both the minimal C
library (Section~\ref{libc}) and the \freebsd{} C library
(Section~\ref{freebsd-libc})
rely on the \posix{} library to provide the necessary
system level operations. For example, \texttt{fopen} in the C library will
chain to \texttt{open} in the \posix{} library, which in turn will chain to
the appropriate {\tt oskit_dir} and {\tt oskit_file} COM operations. All of
the pathname operations, file descriptor bookkeeping, locking, and other
details normally carried out in a ``kernel'' implementation of a system
call interface, are handled by the \posix{} library. Alternatively, the
\posix{} library bridges differences between the COM interfaces and the
functions as defined by \posix{}. 

Since almost all of the functions and definitions provided by the \posix{}
library implement well-known, well-defined ANSI and POSIX C library
interfaces which are amply documented elsewhere, we do not attempt to
describe the purpose and behavior of each function in this chapter.
Instead, only specfic peculiarities, such as implementation
interdependencies and side effects, are described here.

The following set of functions are implemented, and correspond to
their POSIX.1 equivalents: accept, access, bind, chdir, chmod, chown,
chroot, close, connect, creat, dup, dup2, fchdir, fchmod, fchown,
fcntl, fpathconf, fstat, fsync, ftruncate, getpagesize, getpeername,
getsockname, getsockopt, gettimeofday, getumask, ioctl, lchown, link,
listen, lseek, lstat, mkdir, mkfifo, mknod, mq_close, mq_getattr,
mq_notify, mq_open, mq_receive, mq_send, mq_setattr, mq_unlink, open,
pathconf, pipe, read, readlink, readv, recv, recvfrom, rename, rmdir,
select, sem_close, sem_destroy, sem_getvalue, sem_init, sem_open,
sem_post, sem_trywait, sem_unlink, sem_wait, send, sendto, setitimer,
setsockopt, shutdown, sigaction, socket, socketpair, stat, symlink,
truncate, umask, unlink, uname, utime, utimes, write, and writev.

\section{Modified Functions}

These functions are not fully implemented, and return an error condition if
called: adjtime, getdirentries, sbrk, and flock.

These functions are trivially implemented, and are primarily intended to
satisfy linktime dependencies (for example, getuid always returns zero):
geteuid, getuid, seteuid, setuid, getgid, and setgid.

A small number of functions have default implementations that call panic if
they are executed. They are provided in order to saitisfy linktime
dependencies, but should not actually be called. This is sometimes useful
when converting Unix applications into \oskit{} kernels. They are execve,
fork, vfork, wait4, and waitpid.

\subsection{getdtablesize: get descriptor table size}

The \texttt{getdtablesize} function returns a constant value, even though
there is no limit on the number of open file descriptors. This function is
provided for backwards comptability with older BSD system call interfaces.

\subsection{mmap, munmap, mprotect: map files into memory}
The default \texttt{mmap} implementation is extremely limited in its
capabilities. Anonymous memory requests are satisfied using malloc. The
combination of \texttt{MAP_PRIVATE} and \texttt{PROT_WRITE} is not
supported. Beyond that, the underlying file or device must provide the
\texttt{oskit_openfile} COM interface. A secondary \texttt{mmap}
implementation is provided when the Simple Virtual Memory module is linked
in. Anonymous memory requests are satisfied with \texttt{svm_alloc} and
\texttt{mprotect} chains to \texttt{svm_protect}. See Section~\ref{svm} for
more details.

\subsection{getpid: get process id}
\texttt{getpid} always returns zero since there is no concept of
``process'' in a standalone \oskit{} application.

\subsection{gettimeofday: get current time}
Timing functions such as \texttt{gettimeofday} are mapped to the (currently
undocumented) \texttt{oskit_clock} COM interface. In essence, this
interface is a very natural adaptation of the \posix{}.1 real-time
extensions.

\apisec{\textnormal{\posix{}} Message Queue and Semaphore}
\begin{apisyn}
  \cinclude{oskit/c/mqueue.h}

  \funcproto mqd_t mq_open(const~char *name, int oflag, ... \unskip);\\
  \funcproto int mq_send(mqd_t mqdes, const~char *msg_ptr,
                         oskit_size_t msg_len, unsigned~int msg_prio);\\
  \funcproto int mq_receive(mqd_t mqdes, char *msg_ptr, oskit_size_t msg_len, unsigned~int *msg_prio);\\
  \funcproto int mq_close(mqd_t mqdes);\\
  \funcproto int mq_unlink(const~char *name);\\
  \funcproto int mq_setattr(mqd_t mqdes, const~struct~mq_attr *mqstat, struct~mq_attr *omqstat);\\
  \funcproto int mq_getattr(mqd_t mqdes, struct~mq_attr *mqstat);\\
  \funcproto int mq_notify(mqd_t mqdes, const~struct~sigevent *notification);
\end{apisyn}

\begin{apidesc}
  The implementation of \posix{} message queue depends on
  the pthread library and cannot be used in single threaded environment.

  The message queue name space is dependent of file system name space
  and semaphore name space.  Message queue descriptors are not related
  with file descriptors.  

  The mq_send() and mq_receive() are cancellation point.

  mq_open()'s 3rd argument (mode_t) is ignored.

\end{apidesc}

\begin{apisyn}
  \cinclude{oskit/c/semaphore.h}

  \funcproto int sem_init(sem_t *sem, int pshared, unsigned~int value);\\
  \funcproto int sem_destroy(sem_t *sem);\\
  \funcproto sem_t *sem_open(const~char *name, int oflag, ... \unskip);\\
  \funcproto int sem_close(sem_t *sem);\\
  \funcproto int sem_unlink(const char *name);\\
  \funcproto int sem_getvalue(sem_t *sem, int *sval);\\
  \funcproto int sem_wait(sem_t *sem);\\
  \funcproto int sem_trywait(sem_t *sem);\\
  \funcproto int sem_post(sem_t *sem);
\end{apisyn}

\begin{apidesc}
  The implementation of \posix{} semaphore depends on
  the pthread library and cannot be used in single threaded environment.

  The semaphore name space is dependent of file system name
  space and message queue name space.
\end{apidesc}

\section{\textnormal{\posix{}} Signals}
\label{posix-signals}

The \posix{} signal interface has been implemented as best as possible,
given the limitations of the \oskit{} environment. In fact, the multi
threaded version of the \posix/\freebsd{} library provides much better
functionality than the single threaded version. As described in
Section~\ref{freebsd-signals}, application programs can make use of some of the
signal handling mechanisms contained in POSIX.1 specification. The
functions that are implemented are \texttt{signal}, \texttt{sigaction},
\texttt{sigprocmask}, \texttt{raise}, \texttt{kill} (which just calls
raise), as well as the compatibilty functions \texttt{sigblock} and
\texttt{sigsetmask}. Additionally, the multi threaded version of the
library implements \texttt{sigwait}, \texttt{sigwaitinfo}, and
\texttt{sigtimedwait}. The pthread specific functions \texttt{pthread_kill}
and \texttt{pthread_sigmask} are implemented in the pthreads library (see
Section~\ref{pthread}).

\apisec{Client Operating System Dependencies}

The \posix{} library (and by extension the C library) derive a number of
its external interfaces from the client OS (see Section~\ref{clientos}).
Rather than relying on linktime dependencies for these interfaces, the
\posix{} library uses a services database (see Section~\ref{oskit-services})
that was provided to the C library when it was
initialized. This services database is pre-loaded (by the client OS when
the kernel boots) with certain COM objects that make the \posix{} library
functional. One such interface object is the \texttt{oskit_libcenv_t} COM
interface, which provides hooks that are very specific to the \posix{} and C
libraries. This, and other interfaces, are looked up in the database as
required. For example, the \emph{filesytem namespace} object is
provided by the \texttt{oskit_libcenv_t} COM object. The first file access
will result in a lookup in the services database for the
\texttt{oskit_libcenv_t} object, and then a call to
\texttt{oskit_libcenv_getfsnamespace} to get the filesystem namespace
COM object (see Section~\ref{fsnamespace}). If the kernel has been
constructed with a filesystem, and the appropriate
client OS initialization done, the call to \texttt{getfsnamespace} will
return sucessfully. Otherwise, \texttt{getfsnamespace} will fail and all
subsequent file operations will fail.

The intent of this ``indirection'' is to reduce (eventually to zero) the
number of linktime dependencies between the C/Posix library and the rest of
the \oskit{} libraries. Not only does this simplify the process of
constructing an oskit kernel, but also makes it possible to do other
interesting things such dynamic composition of services.

The \posix{} library depends on the client OS for a number of interfaces.
They are:

\subsection{_exit}
\label{oskit-libc-exit}
The \texttt{_exit} call, which terminates the calling process in Unix,
indirects through the \texttt{oskit_libcenv_t} COM object to the client OS
defined reboot mechansim. While the exit status code is passed along, it is
generally ignored. 

\subsection{Console Stream}
The console stream satisfies reads and writes (among others) to and from
the three lowest numbered file descriptors, which are typically the
application's stdin, stdout, and stderr descriptors. The client OS needs to
provide a console stream COM object in order for the application to
interact with the console.

\subsection{Filesystem Namespace}
As described above, access to the filesystem namespace is is provided by
the client OS\@. The \posix{} library uses the namespace COM object to
translate multi component pathnames (paths with '/' in them) into
\texttt{oskit_file} and \texttt{oskit_dir} references.  The namespace
object also provides symbolic link translation and mountpoint traversal
(see Section~\ref{fsnamespace}).

\subsection{System Clock}
The ``system clock'' is an \texttt{oskit_clock} COM object that provides
time of day functions and periodic timer support, such as required by
\texttt{gettimeofday}, \texttt{setitimer} and \texttt{select}.

\subsection{Socket Factory}
The \texttt{socket} and \texttt{socketpair} calls require the
\texttt{oskit_socket_factory_t} COM interface to create sockets and
socketpairs. The socketfactory is created when the clientos initializes the
network stack during system initialization.

\apisec{Extended API functions}

%%%%%%%%%%%%%%%%%%%%%%%%%%%%%%%%%%%%%%%%%%%%%%%%%%%%%%%%%%%%%%%%%%%%%%%%%%%%
%	fs_mount
%
\api{fs_mount, fs_unmount}{Compose file system name spaces}
\label{fs-mount}
\label{fs-unmount}
\begin{apisyn}
        \cinclude{oskit/c/fs.h}

        \funcproto oskit_error_t 
		fs_mount(const~char *path, oskit_file_t *subtree); \\
	\funcproto oskit_error_t fs_unmount(const~char *path);
\end{apisyn}
\begin{apidesc}
	BSD-like mount and unmount functions which the client can use
	to build its file system namespace out of multiple file systems.
	
	Note that the underlying \texttt{oskit_dir} COM interface 
	doesn't support mount points, so crossing mount points while
	traversing the file system space is implemented in the filesystem
	namespace COM object (see Section~\ref{fsnamespace}).
\end{apidesc}
\begin{apiparm}
	\item[path] A valid pathname in the current file system space
		where the file system should be added or removed.
	\item[subtree] The root of the file system to be added.
		A reference to \emph{subtree} is acquired.
\end{apiparm}
\begin{apiret}
	Returns zero on success, or an appropriate error code.
\end{apiret}
