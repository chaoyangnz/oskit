%
% Copyright (c) 1997-2000 University of Utah and the Flux Group.
% All rights reserved.
% 
% The University of Utah grants you the right to copy and reproduce this
% document or portions thereof for academic, research, evaluation, and
% personal use only, provided that (1) the title page appears prominently,
% and (2) these copyright and permission notices are retained in all copies.
% To arrange for alternate terms, contact the University of Utah at
% csl-dist@cs.utah.edu or +1-801-585-3271.
%
% -*- LaTeX -*-

\label{fs}

\section{Introduction}

	The \oskit{} file system framework has parallel goals to the \oskit{}
device driver framework; the framework provides a file system
interface specification designed to allow existing filesystem
implementations to be borrowed from well-established operating systems
in source form and used mostly unchanged to provide file system
support in new operating systems.  The framework is also designed to
allow file systems to be implemented in diverse ways and then composed
together.

	The \oskit{} file system framework encompasses a collection of COM
interfaces used by the client operating system to invoke the file
system libraries, and a collection of interfaces used by the file
system libraries to request services from the client operating system.
The individual file system libraries supply additional interfaces to
the client operating system for initialization, and may supply
additional interfaces for supporting extended features unique to
particular file system implementations.

	The \oskit{} File,
Directory and Open File COM interfaces inherit from several general COM
interfaces, such as Stream, Absolute IO and POSIX IO\@.  The inheritance
relationships among these COM interfaces are shown in Figure~\ref{fig-fs-hier}.
Refer to Section~\ref{com} for more details on COM interfaces.

\psfigure{fs-hier}{%
Interface hierarchy.
Solid lines indicate that the child
interface directly inherits the methods of the parent interface.
Dashed lines indicate that the child interface may optionally support
the parent interface; this may be determined by querying the object.}

\apiintf{oskit_principal}{Principal Interface}

	The {\tt oskit_principal} COM interface defines an interface for
obtaining identity information about a principal (aka subject or
client).  The filesystem libraries obtain an {\tt oskit_principal} object for
the current client by invoking {\tt oskit_get_call_context} on
{\tt oskit_principal_iid}.

	The {\tt oskit_principal} COM interface inherits from {\tt IUnknown},
and has one additional method:
\begin{icsymlist}
\item[getid]
	Obtain identity attributes of principal.
\end{icsymlist}

\api{getid}{Get the identity attributes of this principal}
\begin{apisyn}
	\cinclude{oskit/principal.h}

	\funcproto oskit_error_t
	oskit_principal_getid(oskit_principal_t *p,
			     \outparam oskit_identity_t *out_id);
\end{apisyn}
\fstoos
\begin{apidesc}
	This method returns the identity attributes of this principal.
	{\tt out_id} is a pointer to an {\tt oskit_identity_t} structure defined
	as follows:
	\cstruct{oskit_identity}{
	    	oskit_uid_t uid;		/* effective user id */
		oskit_gid_t gid;		/* effective group id */
		oskit_u32_t ngroups;		/* number of groups */
		oskit_u32_t *groups;		/* supplemental groups */
	};
\end{apidesc}
\begin{apiparm}
	\item[p]
		The principal whose identity is desired.
	\item[out_id]
		The identity attributes of this principal.
\end{apiparm}
\begin{apiret}
	Returns 0 on success, or an error code specified in
	{\tt <oskit/error.h>}, on error.
\end{apiret}


\apiintf{oskit_filesystem}{File System Interface}

The {\tt oskit_filesystem} COM interface defines an interface for
operating on a filesystem, which is a logical collection of files and
a tree-structured namespace with a single root.  The filesystem
itself exists independent of any given namespace,
for there is no notion of mounts in this interface.
That functionality must be implemented at a higher level.

The {\tt oskit_filesystem} COM interface inherits from {\tt IUnknown},
and has the following additional methods:	
\begin{icsymlist}
\item[statfs]
	Get attributes of this filesystem
\item[sync]	
	Write this filesystem's data to permanent storage
\item[getroot]
	Get this filesystem's root directory
\item[remount]
	Update this filesystem's mount flags 
\item[unmount]
	Forcibly unmount this filesystem
\item[lookupi]
	Lookup a file by inode number
\end{icsymlist}

\api{statfs}{Get attributes of this filesystem}
\begin{apisyn}
	\cinclude{oskit/fs/filesystem.h}

	\funcproto oskit_error_t
	oskit_filesystem_statfs(oskit_filesystem_t *f, 
			       \outparam oskit_statfs_t *out_stats);
\end{apisyn}
\ostofs
\begin{apidesc}
	This method returns the attributes of this filesystem.
	{\tt out_stats} is a pointer to an {\tt oskit_statfs_t} structure 
	defined	as follows:
	\cstruct{oskit_statfs}{
	    oskit_u32_t bsize;   /* file system block size */
	    oskit_u32_t frsize;  /* fundamental file system block size */
	    oskit_u32_t blocks;  /* total blocks in fs in units of frsize */
	    oskit_u32_t bfree;   /* free blocks in fs */
	    oskit_u32_t bavail;  /* free blocks avail to non-superuser */
	    oskit_u32_t files;   /* total file nodes in file system */
	    oskit_u32_t ffree;   /* free file nodes in fs */
	    oskit_u32_t favail;  /* free file nodes avail to non-superuser */
	    oskit_u32_t fsid;    /* file system id */
	    oskit_u32_t flag;    /* mount flags */	
	    oskit_u32_t namemax; /* max bytes in a file name */
	};
\end{apidesc}
\begin{apiparm}
	\item[f]
		The filesystem whose attributes are desired.
	\item[out_stats]
		The attributes of the specified filesystem.
\end{apiparm}
\begin{apiret}
	Returns 0 on success, or an error code specified in
	{\tt <oskit/error.h>}, on error.
\end{apiret}


\api{sync}{Synchronize in-core filesystem data with permanent storage}
\begin{apisyn}
	\cinclude{oskit/fs/filesystem.h}

	\funcproto oskit_error_t
	oskit_filesystem_sync(oskit_filesystem_t *f, 
			      oskit_bool_t wait); 	
\end{apisyn}
\ostofs
\begin{apidesc}
	This method writes all of this filesystem's data back to permanent
	storage.  If {\tt wait} is {\tt TRUE}, then the call does not return
	until all pending data has been completely written.

\end{apidesc}
\begin{apiparm}
	\item[f]
		The filesystem to sync.
	\item[wait]
		{\tt TRUE} if the call should wait for completion.
\end{apiparm}
\begin{apiret}
	Returns 0 on success, or an error code specified in
	{\tt <oskit/error.h>}, on error.
\end{apiret}


\api{getroot}{Return a reference to the root directory of this filesystem}
\begin{apisyn}
	\cinclude{oskit/fs/filesystem.h}

	\funcproto oskit_error_t
	oskit_filesystem_getroot(oskit_filesystem_t *f, 
			       \outparam oskit_dir_t **out_dir);
\end{apisyn}
\ostofs
\begin{apidesc}
	This method returns a reference to the root directory of
	this filesystem.  {\tt out_dir} is a pointer to the
	{\tt oskit_dir} COM interface for the root directory.
\end{apidesc}
\begin{apiparm}
	\item[f]
		The filesystem whose root directory is desired.
	\item[out_dir]
		The {\tt oskit_dir} COM interface for the root directory.
\end{apiparm}
\begin{apiret}
	Returns 0 on success, or an error code specified in
	{\tt <oskit/error.h>}, on error.
\end{apiret}


\api{remount}{Update the mount flags of this filesystem}
\begin{apisyn}
	\cinclude{oskit/fs/filesystem.h}

	\funcproto oskit_error_t
	oskit_filesystem_remount(oskit_filesystem_t *f, 
				 oskit_u32_t flags);
\end{apisyn}
\ostofs
\begin{apidesc}
	This method changes the mount flags associated
	with this filesystem.  For example,
	this method might be used to change a filesystem
	from read-only to read-write, or vice versa.
\end{apidesc}
\begin{apiparm}
	\item[f]
		The filesystem whose flags are to be changed.
	\item[flags]
		The new mount flags value.
\end{apiparm}
\begin{apiret}
	Returns 0 on success, or an error code specified in
	{\tt <oskit/error.h>}, on error.
\end{apiret}


\api{unmount}{Forcibly unmount this filesystem}
\begin{apisyn}
	\cinclude{oskit/fs/filesystem.h}

	\funcproto oskit_error_t
	oskit_filesystem_unmount(oskit_filesystem_t *f); 
\end{apisyn}
\ostofs
\begin{apidesc}
	This method forcibly unmounts this
	filesystem.  Ordinarily, a filesystem is
	unmounted when the last reference to it is
	released; in contrast, this method forces
	an unmount regardless of external references
	to the filesystem, and is consequently unsafe.
	Subsequent attempts to use references to the
	filesystem or to use references to files within
	the filesystem may yield undefined results.
\end{apidesc}
\begin{apiparm}
	\item[f]
		The filesystem to be forcibly unmounted.
\end{apiparm}
\begin{apiret}
	Returns 0 on success, or an error code specified in
	{\tt <oskit/error.h>}, on error.
\end{apiret}


\api{lookupi}{Lookup a file by inode number}
\begin{apisyn}
	\cinclude{oskit/fs/filesystem.h}

	\funcproto oskit_error_t
	oskit_filesystem_lookupi(oskit_filesystem_t *f, oskit_ino_t ino,
				\outparam oskit_file_t **out_file);
\end{apisyn}
\ostofs
\begin{apidesc}
	This method looks up a file given its inode number.
	If the inode number is invalid,
	the behavior is undefined.
\end{apidesc}
\begin{apiparm}
	\item[f]
		The filesystem to find the inode in.

	\item[ino]
		The inode number of the file to find.

	\item[out_file]
		Upon success, will point to a oskit_file_t for the file.
\end{apiparm}
\begin{apiret}
	Returns 0 on success, or an error code specified in
	{\tt <oskit/error.h>}, on error.
\end{apiret}


\apiintf{oskit_file}{File Interface}
\label{oskit-file-intf}

	The {\tt oskit_file} COM interface defines an interface for operating
on a file.  The interface does not imply any per-open state; per-open
methods are defined by the {\tt oskit_openfile} COM interface.

	The {\tt oskit_file} COM interface inherits from the {\tt oskit_posixio} 
COM interface, and has the following additional methods:
\begin{icsymlist}
\item[sync]
	Write this file's data and metadata to permanent storage.
\item[datasync]
	Write this file's data to permanent storage.
\item[access]
	Check accessibility of this file.
\item[readlink]
	Read the contents of this symbolic link.
\item[open]
	Create an open instance of this file.
\item[getfs]
	Get the filesystem in which this file resides.
\end{icsymlist}

	Additionally, an {\tt oskit_file} object may export a {\tt
oskit_absio} COM interface; this may be determined by querying the
object.

\api{sync}{Write this file's data and metadata to permanent storage}
\begin{apisyn}
	\cinclude{oskit/fs/file.h}

	\funcproto oskit_error_t
	oskit_file_sync(oskit_file_t *f, oskit_bool_t wait); 
\end{apisyn}
\ostofs
\begin{apidesc}
	This method synchronizes the in-core copy of this
	file's data and metadata with the on-disk copy.
	If {\tt wait} is {\tt TRUE}, then the call does not return
	until all pending data has been completely written.
\end{apidesc}
\begin{apiparm}
	\item[f]
		The file to sync.
	\item[wait]
		{\tt TRUE} if the call should wait for completion.
\end{apiparm}
\begin{apiret}
	Returns 0 on success, or an error code specified in
	{\tt <oskit/error.h>}, on error.
\end{apiret}


\api{datasync}{Write this file's data to permanent storage}
\begin{apisyn}
	\cinclude{oskit/fs/file.h}

	\funcproto oskit_error_t
	oskit_file_datasync(oskit_file_t *f, oskit_bool_t wait); 
\end{apisyn}
\ostofs
\begin{apidesc}
	This method synchronizes the in-core copy of this
	file's data with the on-disk copy.  The file
	metadata need not be sychronized by this method.
	If {\tt wait} is {\tt TRUE}, then the call does not return
	until all pending data has been completely written.
\end{apidesc}
\begin{apiparm}
	\item[f]
		The file to sync.
	\item[wait]
		{\tt TRUE} if the call should wait for completion.
\end{apiparm}
\begin{apiret}
	Returns 0 on success, or an error code specified in
	{\tt <oskit/error.h>}, on error.
\end{apiret}

\api{access}{Check accessibility of this file}
\begin{apisyn}
	\cinclude{oskit/fs/file.h}

	\funcproto oskit_error_t
	oskit_file_access(oskit_file_t *f, oskit_amode_t mask); 
\end{apisyn}
\ostofs
\begin{apidesc}
	This method checks whether the form of access
	specified by {\tt mask} would be granted. 	
	{\tt mask} may be any combination of {\tt OSKIT_R_OK} (read access),
	{\tt OSKIT_W_OK} (write access), or {\tt OSKIT_X_OK} (execute access).
	If the access would not be granted, then this method will return
	the error that would be returned if the	actual access were attempted.
\end{apidesc}
\begin{apiparm}
	\item[f]
		The file whose accessibility is to be checked.
	\item[mask]
		The access mask.
\end{apiparm}
\begin{apiret}
	Returns 0 on success, or an error code specified in
	{\tt <oskit/error.h>}, on error.
\end{apiret}


\api{readlink}{Read the contents of this symbolic link}
\begin{apisyn}
	\cinclude{oskit/fs/dir.h}

	\funcproto oskit_error_t
	oskit_file_readlink(oskit_file_t *f, char *buf, oskit_u32_t len,
			  \outparam oskit_u32_t *out_actual);
\end{apisyn}
\ostofs
\begin{apidesc}
	If this file is a symbolic link, then this method
	reads the contents of the symbolic link into {\tt buf}.	
	No more than {\tt len} bytes will be read.  {\tt out_actual} 
	will be set to the actual number of bytes read.
\end{apidesc}
\begin{apiparm}
	\item[f]
		The symbolic link file.
	\item[buf]
		The buffer into which the contents are to be copied.
	\item[len]
		The maximum number of bytes to read.
	\item[out_actual]
		The actual bytes read from the symlink.
\end{apiparm}
\begin{apiret}
	Returns 0 on success, or an error code specified in
	{\tt <oskit/error.h>}, on error.

	If the file is not a symbolic link, then {\tt OSKIT_E_NOTIMPL}
	is returned.
\end{apiret}


\api{open}{Create an open instance of this file}
\label{fs-file-open}
\begin{apisyn}
	\cinclude{oskit/fs/file.h}
	\\
	\cinclude{oskit/fs/openfile.h}

	\funcproto oskit_error_t
	oskit_file_open(oskit_file_t *f, 
	               oskit_oflags_t flags,
		       \outparam oskit_openfile_t **out_openfile);
\end{apisyn}
\ostofs
\begin{apidesc}
	This method returns an {\tt oskit_openfile} COM interface
	for an open instance of this file.
	{\tt flags } specifies the file open flags, as
	defined in {\tt <oskit/fs/file.h>}.  If {\tt OSKIT_O_TRUNC}
	is specified, then the file will be truncated
	to zero length. 

	This method may only be used on regular 	
	files and directories.   Directories
	may not be opened with {\tt OSKIT_O_WRONLY},
	{\tt OSKIT_O_RDWR} or {\tt OSKIT_O_TRUNC}.

	This method may return success but set
	{\tt *out_openfile} to {\tt NULL}, indicating that
	the requested operation is allowed but the
	filesystem does not support per-open state;
	the client operating system must provide this
	functionality.
\end{apidesc}
\begin{apiparm}
	\item[f]
		The file to open.
	\item[flags]
		The open flags.
	\item[out_openfile]	
		The {\tt oskit_openfile} COM interface.
\end{apiparm}
\begin{apiret}
	Returns 0 on success, or an error code specified in
	{\tt <oskit/error.h>}, on error.
\end{apiret}


\api{getfs}{Get the filesystem in which this file resides}
\begin{apisyn}
	\cinclude{oskit/fs/file.h}

	\funcproto oskit_error_t
	oskit_file_getfs(oskit_file_t *f, 
	                \outparam oskit_filesystem_t **out_fs);
\end{apisyn}
\ostofs
\begin{apidesc}
	Returns the {\tt oskit_filesystem} COM interface for
	the filesystem in which this file resides.
\end{apidesc}
\begin{apiparm}
	\item[f]
		The file whose filesystem is desired.
	\item[out_fs]
		The filesystem in which the file resides.
\end{apiparm}
\begin{apiret}
	Returns 0 on success, or an error code specified in
	{\tt <oskit/error.h>}, on error.
\end{apiret}


\apiintf{oskit_dir}{Directory Interface}

	The {\tt oskit_dir} COM interface defines an interface for operating
on a directory.  The interface does not imply any per-open state; per-open
methods are defined by the {\tt oskit_openfile} COM interface.

	The {\tt oskit_dir} COM interface inherits from the {\tt oskit_file} COM
interface, and has the following additional methods:
\begin{csymlist}
\item[lookup]
	Lookup a file in this directory.
\item[create]
	Create a regular file in this directory.
\item[link]
	Link a file into this directory.
\item[unlink]
	Unlink a file from this directory.
\item[rename]
	Rename a file from this directory.
\item[mkdir]
	Create a directory in this directory.
\item[rmdir]
	Remove a directory from this directory.
\item[getdirentries]
	Read entries from this directory.
\item[mknod]
	Create a special file in this directory.
\item[symlink]
	Create a symlink in this directory.
\item[reparent]
	Create a virtual directory from this directory.
\end{csymlist}

	Additionally, an {\tt oskit_dir} object may export a {\tt
oskit_absio} COM interface; this may be determined by querying the
object.

	All name parameters to directory methods must be a single
component, ie an entry in one of the specified directories.  With the
exception of {\tt rename}, name parameters always refer to entries in
the target directory itself.

\api{lookup}{Look up a file in this directory}
\begin{apisyn}
	\cinclude{oskit/fs/dir.h}

	\funcproto oskit_error_t
	oskit_dir_lookup(oskit_dir_t *d, const~char *name, 
			\outparam oskit_file_t **out_file); 
\end{apisyn}
\ostofs
\begin{apidesc}
	This method returns the {\tt oskit_file} COM interface
	for the file named by {\tt name} in this directory.
	The {\tt name} may only be a single component;
	multi-component lookups are not supported.
	If the file is a symbolic link, then {\tt out_file}
	will reference the symbolic link itself.
\end{apidesc}
\begin{apiparm}
	\item[d]
		The directory to search.
	\item[name]
		The name of the file.
	\item[out_file]
		The {\tt oskit_file} COM interface for the file.
\end{apiparm}
\begin{apiret}
	Returns 0 on success, or an error code specified in
	{\tt <oskit/error.h>}, on error.
\end{apiret}


\api{create}{Create a regular file in this directory}
\begin{apisyn}
	\cinclude{oskit/fs/dir.h}

	\funcproto oskit_error_t
	oskit_dir_create(oskit_dir_t *d, const~char *name, 
			oskit_bool_t excl, oskit_mode_t mode,
			\outparam oskit_file_t **out_file); 
\end{apisyn}
\ostofs
\begin{apidesc}
	This method is the same as {\tt oskit_dir_lookup}, except
	that if the file does not exist, then a regular file
	will be created with the specified {\tt name} and
	{\tt mode}.

	If a file with {\tt name} already exists, and 	
	{\tt excl} is {\tt TRUE}, then {\tt OSKIT_EEXIST} will be
	returned.

	The {\tt name} may only be a single component;
	multi-component lookups are not supported.
\end{apidesc}
\begin{apiparm}
	\item[d]
		The directory to search.
	\item[name]
		The name of the file.
	\item[excl]
		{\tt TRUE} if an error should be returned if the file exists
	\item[mode]
		The file mode to use if creating a new file.
	\item[out_file]
		The {\tt oskit_file} COM interface for the file.
\end{apiparm}
\begin{apiret}
	Returns 0 on success, or an error code specified in
	{\tt <oskit/error.h>}, on error.
\end{apiret}


\api{link}{Link a file into this directory}
\begin{apisyn}
	\cinclude{oskit/fs/dir.h}

	\funcproto oskit_error_t
	oskit_dir_link(oskit_dir_t *d, const~char *name, 
	              oskit_file_t *file); 
\end{apisyn}
\ostofs
\begin{apidesc}
	This method adds an entry for {\tt file} into this directory,
	using {\tt name} for the new directory entry.
	Typically, this is only supported if {\tt file}
	resides in the same filesystem as {\tt d}.

	{\tt file} may not be a symbolic link.  

	The {\tt name} may only be a single component;
	multi-component lookups are not supported.
\end{apidesc}
\begin{apiparm}
	\item[d]
		The directory to search.
	\item[name]
		The name for the new link.
	\item[file]
		The file to be linked.
\end{apiparm}
\begin{apiret}
	Returns 0 on success, or an error code specified in
	{\tt <oskit/error.h>}, on error.
\end{apiret}


\api{unlink}{Unlink a file from this directory}
\begin{apisyn}
	\cinclude{oskit/fs/dir.h}

	\funcproto oskit_error_t
	oskit_dir_unlink(oskit_dir_t *d, const~char *name); 
\end{apisyn}
\ostofs
\begin{apidesc}
	This method removes the directory entry for {\tt name}
	from {\tt d}.

	The {\tt name} may only be a single component;
	multi-component lookups are not supported.
\end{apidesc}
\begin{apiparm}
	\item[d]
		The directory to search.
	\item[name]
		The name of the file to be unlinked.
\end{apiparm}
\begin{apiret}
	Returns 0 on success, or an error code specified in
	{\tt <oskit/error.h>}, on error.
\end{apiret}


\api{rename}{Rename a file from this directory}
\begin{apisyn}
	\cinclude{oskit/fs/dir.h}

	\funcproto oskit_error_t
	oskit_dir_rename(oskit_dir_t *old_dir, const~char *old_name,
			 oskit_dir_t *new_dir, const~char *new_name); 
\end{apisyn}
\ostofs
\begin{apidesc}
	This method atomically links the file named by {\tt old_name} in 
	{\tt old_dir} into {\tt new_dir}, using {\tt new_name} for the
	new directory entry, and unlinks {\tt old_name} from {\tt old_dir}. 

	If a file named {\tt new_name} already exists in {\tt new_dir}, 
	then it is first removed.  In this case, the source and target files
	must either both be directories or both be non-directories,
	and if the target file is a directory, it must be empty.

	Typically, this is only supported if {\tt new_dir}
	resides in the same filesystem as {\tt old_dir}.

	The {\tt old_name} and {\tt new_name} may each only be 
	a single component; multi-component lookups are not supported.
\end{apidesc}
\begin{apiparm}
	\item[old_dir]
		This directory.
	\item[old_name]
		The name of the file to be renamed.
	\item[new_dir]
		The target directory.
	\item[new_name]
		The name for the new directory entry.
\end{apiparm}
\begin{apiret}
	Returns 0 on success, or an error code specified in
	{\tt <oskit/error.h>}, on error.
\end{apiret}


\api{mkdir}{Create a subdirectory in this directory}
\begin{apisyn}
	\cinclude{oskit/fs/dir.h}

	\funcproto oskit_error_t
	oskit_dir_mkdir(oskit_dir_t *d, const~char *name,	
		       oskit_mode_t mode);	
\end{apisyn}
\ostofs
\begin{apidesc}
	This method creates a new subdirectory in this directory,
	with the specified {\tt name} and {\tt mode}.

	The {\tt name} may only be a single component; 
	multi-component lookups are not supported.
\end{apidesc}
\begin{apiparm}
	\item[dir]
		The directory in which to create the subdirectory.
	\item[name]
		The name of the new subdirectory.
	\item[mode]
		The mode for the new subdirectory.
\end{apiparm}
\begin{apiret}
	Returns 0 on success, or an error code specified in
	{\tt <oskit/error.h>}, on error.
\end{apiret}


\api{rmdir}{Remove a subdirectory from this directory}
\begin{apisyn}
	\cinclude{oskit/fs/dir.h}

	\funcproto oskit_error_t
	oskit_dir_rmdir(oskit_dir_t *d, const~char *name);	
\end{apisyn}
\ostofs
\begin{apidesc}
	This method removes the subdirectory named {\tt name} 
	from this directory.  Typically, this is only
	supported if the subdirectory is empty.

	The {\tt name} may only be a single component; 
	multi-component lookups are not supported.
\end{apidesc}
\begin{apiparm}
	\item[dir]
		The directory in which the subdirectory resides.
	\item[name]
		The name of the subdirectory.
\end{apiparm}
\begin{apiret}
	Returns 0 on success, or an error code specified in
	{\tt <oskit/error.h>}, on error.
\end{apiret}


\api{getdirentries}{Read one or more entries from this directory}
\begin{apisyn}
	\cinclude{oskit/fs/dir.h}
	\cinclude{oskit/fs/dirents.h}

	\funcproto oskit_error_t
	oskit_dir_getdirentries(oskit_dir_t *d, 
				oskit_u32_t *inout_ofs,
				oskit_u32_t nentries,
				\outparam oskit_dirents_t **out_dirents);
\end{apisyn}
\ostofs
\begin{apidesc}
	This method reads one or more entries from this
	directory.  On entry, {\tt inout_ofs} contains the	
	offset of the first entry to be read.  Before
	returning, this method updates the value at 
	{\tt inout_ofs} to contain the offset of the
	next entry after the last entry	returned in
	{\tt out_dirents}.  The returned value of 
	{\tt inout_ofs} is opaque; it should only be used 
	in subsequent calls to this method.

	This method will return at least {\tt nentries}
	entries if there are at least that many entries
	remaining in the directory; however, this method
	may return more entries.

	The return value {\tt out_dirents} will contain a pointer to
	an {\tt oskit_dirents_t} COM object, which holds the individual
	directory entries. The number of actual entries returned can be
	determined with the {\tt oskit_dirents_getcount} method. Each
	successive directory entry is accessed using the
	{\tt oskit_dirents_getnext} method. Once all the entries have
	been read, the dirents COM object should be released with the
	{\tt oskit_dirents_release} method.

	The data structure to retrieve the individual entries is:
	\cstruct{oskit_dirent}{
		oskit_size_t namelen;		/* name length */
		oskit_ino_t ino;		/* entry inode */
		char name[0];			/* entry name */
	};

	The {\tt namelen} field should be initialized to the amount of
	storage available for the name. Upon return from the getnext
	method, {\tt namelen} will be set to the actual length of the
	name. The pointer that is passed should obviously be large
	enough to hold the size of the structure above, plus the
	additional size of the character array.

\end{apidesc}
\begin{apiparm}
	\item[d]
		The directory to read.
	\item[inout_ofs]
		On entry, the offset of the first entry to read.
		On exit, the offset of the next entry to read.
	\item[nentries]
		The minimum desired number of entries.
	\item[out_dirents]
		The directory entries oskit_dirents_t COM object.
\end{apiparm}
\begin{apiret}
	Returns 0 on success, or an error code specified in
	{\tt <oskit/error.h>}, on error.
\end{apiret}


\api{mknod}{Create a special file node in this directory}
\begin{apisyn}
	\cinclude{oskit/fs/dir.h}

	\funcproto oskit_error_t
	oskit_dir_mknod(oskit_dir_t *d, const~char *name,	
		       oskit_mode_t mode, oskit_dev_t dev);	
\end{apisyn}
\ostofs
\begin{apidesc}
	This method creates a device special file in this directory,
	with the specified {\tt name} and file {\tt mode}, 
	and with the specified device number {\tt dev}.  The
	device number is opaque to the filesystem library.

	The {\tt name} may only be a single component; 
	multi-component lookups are not supported.
\end{apidesc}
\begin{apiparm}
	\item[d]
		The directory in which to create the node
	\item[name]
		The name of the new node.
	\item[mode]
		The mode for the new node.
	\item[dev]
		The device number for the new node.
\end{apiparm}
\begin{apiret}
	Returns 0 on success, or an error code specified in
	{\tt <oskit/error.h>}, on error.
\end{apiret}


\api{symlink}{Create a symbolic link in this directory}
\begin{apisyn}
	\cinclude{oskit/fs/dir.h}

	\funcproto oskit_error_t
	oskit_dir_symlink(oskit_dir_t *d, const~char *link_name,	
		         char *dest_name);
\end{apisyn}
\ostofs
\begin{apidesc}
	This method creates a symbolic link in this directory,
	named {\tt link_name}, with contents {\tt dest_name}.

	The {\tt link_name} may only be a single component; 
	multi-component lookups are not supported.
\end{apidesc}
\begin{apiparm}
	\item[d]
		The directory in which to create the symlink.
	\item[link_name]
		The name of the new symlink.
	\item[dest_name]
		The contents of the new symlink.
\end{apiparm}
\begin{apiret}
	Returns 0 on success, or an error code specified in
	{\tt <oskit/error.h>}, on error.
\end{apiret}


\api{reparent}{Create a virtual directory from this directory}
\begin{apisyn}
	\cinclude{oskit/fs/dir.h}

	\funcproto oskit_error_t
	oskit_dir_reparent(oskit_dir_t *d, oskit_dir_t *parent,
			  \outparam oskit_dir_t **out_dir);

\end{apisyn}
\ostofs
\begin{apidesc}
	This method creates a virtual directory {\tt out_dir}
	which refers to the same underlying directory as {\tt d}, 
	but whose logical parent directory is {\tt parent}.
	If {\tt parent} is {\tt NULL}, then the logical parent
	directory of {\tt out_dir} will be itself.

	Lookups of the parent directory entry ('..') in the virtual
	directory will return a reference to the logical parent
	directory.

	This method may be used to provide equivalent
	functionality to the Unix {\tt chroot} operation.
\end{apidesc}
\begin{apiparm}
	\item[d]
		The directory 
	\item[parent]
		The logical parent directory
	\item[out_dir]
		The {\tt oskit_dir} COM interface for the virtual directory
\end{apiparm}
\begin{apiret}
	Returns 0 on success, or an error code specified in
	{\tt <oskit/error.h>}, on error.
\end{apiret}


\apiintf{oskit_openfile}{Open File Interface}

	The {\tt oskit_openfile} COM interface defines an interface for
operating on an open instance of a file.

	The {\tt oskit_openfile} COM interface inherits from the {\tt
oskit_stream} COM interface, and has the following additional method:
\begin{icsymlist}
\item[getfile]
	Get the underlying file object to which this open file refers.
\end{icsymlist}

	Additionally, an {\tt oskit_openfile} object may export a {\tt
oskit_absio} COM interface; this may be determined by querying the
object.

\api{getfile}{Get the underlying file object to which this open file refers}
\begin{apisyn}
	\cinclude{oskit/fs/openfile.h}

	\funcproto oskit_error_t
	oskit_openfile_getfile(oskit_openfile_t *f,
			      \outparam oskit_file_t **out_file);
\end{apisyn}
\ostofs
\begin{apidesc}
	This method returns the {\tt oskit_file} COM interface for
	the underlying file object to which this open file refers.
\end{apidesc}
\begin{apiparm}
	\item[f]
		The open file whose underlying file is desired.
	\item[out_file]
		The {\tt oskit_file} COM interface for the underlying file.
\end{apiparm}
\begin{apiret}
	Returns 0 on success, or an error code specified in
	{\tt <oskit/error.h>}, on error.
\end{apiret}


\apisec{Dependencies on the Client Operating System}

	This section describes the interfaces which must be provided
by the client operating system to the filesystem library.

	These interfaces consist of:
\begin{icsymlist}
\item[oskit_get_call_context]
	Obtain information about client context.
\end{icsymlist}

\api{oskit_get_call_context}{Get the caller's context}
\begin{apisyn}
	\cinclude{oskit/com.h}

	\funcproto oskit_error_t
	oskit_get_call_context(oskit_guid_t *iid,
			      \outparam void **out_if);	
\end{apisyn}
\fstoos
\begin{apidesc}
	This function returns the requested COM interface
	for the current caller.

	Typically, this is used to obtain the {\tt oskit_principal}
	object for the current client of the filesystem library.	
\end{apidesc}
\begin{apiparm}
	\item[iid]
		The desired COM interface identifier.
	\item[out_if]
		The COM interface.
\end{apiparm}
\begin{apiret}
	Returns 0 on success, or an error code specified in
	{\tt <oskit/error.h>}, on error.
\end{apiret}

