%
% Copyright (c) 1997-2000 University of Utah and the Flux Group.
% All rights reserved.
% 
% The University of Utah grants you the right to copy and reproduce this
% document or portions thereof for academic, research, evaluation, and
% personal use only, provided that (1) the title page appears prominently,
% and (2) these copyright and permission notices are retained in all copies.
% To arrange for alternate terms, contact the University of Utah at
% csl-dist@cs.utah.edu or +1-801-585-3271.
%
\label{smp}

\section{Introduction}

This library is designed to simplify the startup and use
of multiprocessors.  It defines a common interface to 
multiprocessor machines that is fairly platform independent.

Combined with the spin-locks provided in {\tt libkern},
it is possible to implement a complete symmetric multiprocessor (SMP)
based system using the \oskit{} code.

There is currently one machine-dependent interface,
{\tt smp_apic_ack} for the x86.

\section{Supported Systems}

Currently, SMP support is only provided for Intel x86
systems conforming to the Intel Multiprocessor Specification.

\subsection{Intel x86}

Systems which fully comply to the Intel MultiProcessing Specification (IMPS)
should be supported.  Since some of the code is based on Linux 2.0,
some features (such as dual I/O APICs) are not fully supported.
The APIC (Advanced Programmable Interrupt Controller)
is not yet used for general interrupt delivery.
Instead, all hardware interrupts are sent to the
BootStrap Processor (BSP).

If a machine works with 
Linux 2.0 it should work with the \oskit; however, testing
has been limited to a few dual-processor machines.

The SMP code must be compiled with a compiler that supports
\emph{.code16} for full functionality.
The smp library will compile without it,
but it will only support a single processor.

Inter-processor interrupts (IPIs) are implemented.  These are currently the
only interrupts received by the Application Processors (APs).  IPIs
allow the client OS to implement TLB-shoot-down and reschedule
requests.

It is important to note that if more than one processor wishes
to run in ``user mode,'' that the per-processor data structures
in {\tt libkern} (such as {\tt base_tss}, {\tt base_idt}, and {\tt base_gdt})
will have to be made per-processor.

The \oskit{} code has not been tested with more than two processors.
Success (and failure) reports for systems with three or more processors
would be appreciated.

{\tt smp_apic_ack} mentions a potential pitfall with Intel x86 SMPs.
If more than one processor tries to send an IPI to a target processor,
or if a processor sends multiple IPIs without waiting for them to
be processed, IPIs can get lost.  It is up to the programmer to
deal with this limitation.

\subsection{External dependencies}

The SMP library assumes that the base environment is usable.
It starts up the Application Processors on the kernel support library's
``base'' data structures.  It is possible (in fact required in many cases)
to reload per-processor copies.

The following are symbols from the kernel support library required by
the SMP library:
\begin{apidep}
	\item[base_gdt]		\S~\ref{base-gdt}
	\item[base_idt]		\S~\ref{base-idt}
	\item[base_tss_load]	\S~\ref{base-tss-load}
	\item[boot_info]	\S~\ref{boot-info}
	\item[phys_mem_va]	\S~\ref{phys-mem-va}
\end{apidep}

The LMM library is used to allocate pages of memory below 1MB\@.
This requires the symbols:
\begin{apidep}
	\item[lmm_alloc_page]	\S~\ref{lmm-alloc-page}
	\item[malloc_lmm]	\S~\ref{malloc-lmm}
\end{apidep}

These minimal C library symbols are pulled in by the SMP support code:
\begin{apidep}
	\item[panic]		\S~\ref{panic}
	\item[printf]		\S~\ref{printf}
\end{apidep}

This library provides SMP-safe implementations for:
\begin{apidep}
	\item[base_critical_enter]	\S~\ref{base-critical-enter}
	\item[base_critical_leave]	\S~\ref{base-critical-leave}
\end{apidep}

\section{API reference}

\api{smp_init}{Initializes the SMP startup code}
\begin{apisyn}
	\cinclude{oskit/smp.h}

        \funcproto int smp_init(void);
\end{apisyn}
\begin{apidesc}
	This function does the initial setup for the SMP support.
	It should be called before any other SMP library routines are
	used.  It identifies the processors and gets them ready
	and waiting in a busy-loop for a ``go'' from the boot
	processor.  

	Note that success \emph{does not} necessarily mean the system
	has multiple processors.  Rather, failure indicates that
	the machine does not support multiple processors.
	{\tt smp_get_num_cpus} should be used to determine the
	number of CPUs present.

	Don't call this more than once\ldots{}yet.
\end{apidesc}
\begin{apiret}
	It returns 0 on success (SMP-capable system is found).
	E_SMP_NO_CONFIG is returned on non-IMPS-compliant x86 machines.
\end{apiret}



\api{smp_find_cur_cpu}{Return the processor ID of the current processor.}
\begin{apisyn}
	\cinclude{oskit/smp.h}

        \funcproto int smp_find_cur_cpu(void);
\end{apisyn}
\begin{apidesc}
	This function returns a unique (per-processor) integer 
	representing the current processor.  Note that the numbers
	are \emph{not} guaranteed to be sequential or starting from 0,
	although that may be a common case.

	On the x86, these numbers correspond to the processor's
	APIC ID, which is set by the hardware.
	However, these are to be treated as logical processor numbers
	since the smp library may do a transformation in the future.
\end{apidesc}
\begin{apiret}
	The processor's ID.
\end{apiret}



\api{smp_find_cpu}{Return the next processor ID}
\begin{apisyn}
	\cinclude{oskit/smp.h}

        \funcproto int smp_find_cpu(int first);
\end{apisyn}
\begin{apidesc}
	Given a number \emph{first}, it returns the first processor ID
	such that the ID is greater than or equal to that number.

	In order to be assured of finding all the CPUs,
	the initial call should be made with an argument of 0
	and subsequent calls should be made with one more than the
	previously returned value.

	This is designed to be used as an iterator function for
	the client OS to determine which processor numbers
	are present.
\end{apidesc}
\begin{apiparm}
        \item[first]
		The processor number at which to start searching.
\end{apiparm}
\begin{apiret}
	Returns E_SMP_NO_PROC if there are no more processors,
	otherwise the ID of the next processor.
\end{apiret}




\api{smp_start_cpu}{Starts a processor running a specified function}
\begin{apisyn}
	\cinclude{oskit/smp.h}

        \funcproto void smp_start_cpu(int processor_id,
				void (*func)(void~*data),
				void *data,
				void *stack_ptr);
\end{apisyn}
\begin{apidesc}
	This releases the specified processor to start running
	a function with the specified stack.

	Results are undefined if:
	\begin{enumerate}
	\item the processor indicated does not exist,
	\item a processor attempts to start itself,
	\item any processor is started more than once, or
	\item any of the parameters is invalid.
	\end{enumerate}

	{\tt smp_find_cur_cpu} can be 
	used to prevent calling {\tt smp_start_cpu} on yourself.
	This function must be called for each processor started
	up by {\tt smp_init}; if the processor is not used,
	then {\tt func} should execute the halt instruction
	immediately.

	It is up to the user to verify that the processor is
	started up correctly.
\end{apidesc}
\begin{apiparm}
	\item[processor_id]
		The ID of a processor found by the startup code.
	\item[func]
		A function pointer to be called by the processor
		after it has set up its stack.
	\item[data]
		A pointer to some structure that is placed on
		that stack before {\tt func} is called.
	\item[stack_ptr]
		The stack pointer to be used by the processor.
		This should point to the top of the stack to be
		used by the processor, and should be large enough
		for {\tt func}'s requirements.
\end{apiparm}



\api{smp_get_num_cpus}{Returns the total number of processors}
\begin{apisyn}
	\cinclude{oskit/smp.h}

        \funcproto int smp_get_num_cpus(void);
\end{apisyn}
\begin{apidesc}
	This returns the number of processors that exist.
\end{apidesc}
\begin{apiret}
	The number of processors that have been found.
	In a non-SMP-capable system, this will always return one.
\end{apiret}


\api{smp_map_range}{Request the OS map physical memory}
\begin{apisyn}
	\cinclude{oskit/smp.h}

        \funcproto oskit_addr_t smp_map_range(oskit_addr_t start,
		oskit_size_t size);
\end{apisyn}
\begin{apidesc}
	This function is a hook provided by the host OS to allow the SMP
	library to request physical memory be mapped into its virtual
	address space.
	This is called by {\tt smp_init_paging}.

	Note that this could be implemented using {\tt osenv_mem_map_phys}.
\end{apidesc}
\begin{apiret}
	The virtual address where the physical pages are mapped.
	Returns zero if unable to map the memory.
\end{apiret}


\api{smp_init_paging}{Tell the SMP code that paging is being enabled}
\begin{apisyn}
	\cinclude{oskit/smp.h}

        \funcproto int smp_init_paging(void);
\end{apisyn}
\begin{apidesc}
	This routine is called by the OS when it is ready to turn on paging.
	This call causes the SMP library to make call-backs to the OS
	to map the regions that are SMP-specific.
	On Intel x86 processors, this means the APICS.
\end{apidesc}
\begin{apiret}
	Zero on success, non-zero on failure.
\end{apiret}


\api{smp_message_pass}{Send an inter-processor interrupt to another CPU}
\begin{apisyn}
	\cinclude{oskit/smp.h}

        \funcproto void smp_message_pass(int cpunum);
\end{apisyn}
\begin{apidesc}
	This causes the target processor to run its interrupt
	handler for the IPI vector, if the appropriate entry of
	{\tt smp_message_pass_enable} has been set to non-zero by
	that processor.  A processor should
	only modify its own {\tt smp_message_pass_enable} entry after
	it is ready to start receiving IPIs.

	This call offers very limited functionality.
	The expectation is that the OS writer will
	implement the desired functionality on top of this primitive.
\end{apidesc}


\api{smp_message_pass_enable}{}
\begin{apisyn} 
        smp_message_pass_enable[CPUID]
\end{apisyn}
\begin{apidesc}
	This array contains an entry for each processor.
	If a processor is ready to start receiving inter-processor
	interrupts, it should set
	smp_message_pass_enable[smp_find_cur_cpu()] to non-zero.
	This is used internally by the SMP library to prevent
	interrupts from being delivered before the processor has
	set up enough state to receive them.
\end{apidesc} 


\api{smp_apic_ack}{\intel\ acknowledge an inter-processor interrupt}
\begin{apisyn}
	\cinclude{oskit/x86/smp.h}

        \funcproto void smp_apic_ack(void);
\end{apisyn}
\begin{apidesc}
	This routine ACKs the local APIC\@.
	The APIC must be ACKed before returning from the IPI handler.
	Due to limitations in the APIC design,
	IPIs can be lost if sent too closely together,
	as the APIC only handles two outstanding requests.
\end{apidesc}

