%
% Copyright (c) 1997-1998 University of Utah and the Flux Group.
% All rights reserved.
% 
% The University of Utah grants you the right to copy and reproduce this
% document or portions thereof for academic, research, evaluation, and
% personal use only, provided that (1) the title page appears prominently,
% and (2) these copyright and permission notices are retained in all copies.
% To arrange for alternate terms, contact the University of Utah at
% csl-dist@cs.utah.edu or +1-801-585-3271.
%
\label{netbsd-fs}

The NetBSD filesystem library consists of the NetBSD filesystem code
and a collection of glue code which encapsulates the NetBSD
code within the \oskit{} filesystem framework.

The NetBSD filesystem library provides two additional interfaces:
\begin{csymlist}
\item[fs_netbsd_init]
	Initialize the NetBSD fs library.
\item[fs_netbsd_mount]
	Mount a filesystem via the NetBSD fs library
\end{csymlist}

\api{fs_netbsd_init}{Initialize the NetBSD fs library}
\begin{apisyn}
	\cinclude{oskit/fs/netbsd.h}

	\funcproto oskit_error_t
		   fs_netbsd_init(void);
\end{apisyn}
\ostofs
\begin{apidesc}
	This function initializes the NetBSD fs library,
	and must be invoked prior to the first
	call to {\tt fs_netbsd_mount}.  This function only
	needs to be invoked once by the client
	operating system.
\end{apidesc}
\begin{apiret}
	Returns 0 on success, or an error code specified in
	{\tt <oskit/error.h>}, on error.
\end{apiret}


\api{fs_netbsd_mount}{Mount a filesystem via the Netbsd fs library}
\begin{apisyn}
	\cinclude{oskit/fs/netbsd.h}

	\funcproto oskit_error_t fs_netbsd_mount(oskit_blkio_t *bio,
						 oskit_u32_t flags,
						 oskit_filesystem_t **out_fs);
\end{apisyn}
\ostofs
\begin{apidesc}
        This function mounts a FFS filesystem from the partition
	described by \texttt{bio},
        and returns a handle to an \texttt{oskit_filesystem_t} for
        this filesystem.

	This function may be used multiple times by a client
	operating system to mount multiple file systems.

	Note that this function does not graft the filesystem
	into a namespace;  {\tt oskit_dir_reparent} or other layers
	may be used for that purpose.

	Typically, this interface is not exported to clients,
	and is only used by the client operating system
	during initialization.
\end{apidesc}
\begin{apiparm}
        \item[bio]
              Describes the partition containing a filesytem.
              Can be the whole disk like that returned from
              \texttt{oskit_linux_block_open},
              or a subset of one like what is given by
              \texttt{diskpart_blkio_lookup_bsd_string}.

        \item[flags]
              The mount flags.
              These are formed by or'ing the following values:
              \begin{description}
                \item[OSKIT_FS_RDONLY] Read only filesystem
                \item[OSKIT_FS_NOEXEC] Can't exec from filesystem
                \item[OSKIT_FS_NOSUID] Don't honor setuid bits on fs
                \item[OSKIT_FS_NODEV]  Don't interpret special files
              \end{description}

        \item[out_fs]
              Upon success, is set to point to an \texttt{oskit_filesystem_t}
              for this filesystem.
\end{apiparm}
\begin{apiret}
	Returns 0 on success, or an error code specified in
	{\tt <oskit/error.h>}, on error.
\end{apiret}
