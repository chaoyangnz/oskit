%
% Copyright (c) 1999, 2000 University of Utah and the Flux Group.
% All rights reserved.
% 
% The University of Utah grants you the right to copy and reproduce this
% document or portions thereof for academic, research, evaluation, and
% personal use only, provided that (1) the title page appears prominently,
% and (2) these copyright and permission notices are retained in all copies.
% To arrange for alternate terms, contact the University of Utah at
% csl-dist@cs.utah.edu or +1-801-585-3271.
%
\label{memfs}

The memfs library contains an implementation of a trivial memory-based 
filesystem.  It exports the standard \oskit{} filesystem interface,
and depends on the underlying osenv interfaces.  (See
Chapter~\ref{fs}).

The header file \texttt{<oskit/fs/memfs.h>} must be included to use
the memfs, and its derivative cousin, the bmodfs.  When first
instantiated, a memfs filesystem is empty, and may be populated using
the standard file access mechanisms.  A special function,
oskit_memfs_file_set_contents is provided for convenience to replace
the contents of a MEMFS filesystem.  Clients who wish to strictly use
the \oskit{} filesystem interface to guarantee portability to other
filesystems should not use this function.


\api{oskit_memfs_init}{initialize MEMFS filesystem}
\begin{apisyn}
	\cinclude{oskit/fs/memfs.h}

	\funcproto oskit_error_t oskit_memfs_init(oskit_osenv_t *osenv,
                                                  oskit_filesystem_t **out_fs);
\end{apisyn}
\begin{apidesc}
	Initialize the MEMFS filesystem.
\end{apidesc}
\begin{apiret}
	Returns handle to an empty MEMFS filesystem.
\end{apiret}

\api{oskit_memfs_file_set_contents}{replace contents of a MEMFS file}
\begin{apisyn}
	\cinclude{oskit/fs/memfs.h}

	\funcproto oskit_error_t
		oskit_memfs_file_set_contents(oskit_file_t *file,
			void *data, oskit_off_t size, oskit_off_t allocsize,
			oskit_bool_t can_sfree, oskit_bool_t inhibit_resize);
\end{apisyn}
\begin{apidesc}
	This function changes the indicated MEMFS file to use the memory
	from [\emph{data} - \emph{data}+\emph{size}-1] as its contents.
	\emph{File} must be a regular MEMFS file and not a directory.

	\emph{Allocsize} indicates the total amount of memory available
	for the file to use when growing and must be greater than or
	equal to \emph{size}.
	If an attempt is made to grow the file to a size greater than
	\emph{allocsize}, new memory will be allocated with
	\texttt{smemalign} and the file contents copied to the new memory.

	If \emph{inhibit_resize} is true, attempts to change the size of
	the file hereafter will fail with \texttt{OSKIT_EPERM}.

	If \emph{can_sfree} is true, \texttt{sfree} is called on the
	data buffer if the file grows beyond \emph{allocsize},
	is truncated to zero-length or is removed.
\end{apidesc}
\begin{apiparm}
	\item[file]
		File in the memfs filesystem whose contents are being replaced.
	\item[data]
		Pointer to memory to be used as the new file contents.
	\item[size]
		The new size of the file.
	\item[allocsize]
		Size of writable memory available for the file.
	\item[can_sfree]
		If true, indicates that the memory pointed to by \emph{data}
		can be released with \texttt{sfree} when the file grows
		beyond \emph{allocsize}, is truncated to zero-length, or
		is removed.
	\item[inhibit_resize]
		If true, fails any attempt to change the size of the file.
\end{apiparm}
\begin{apiret}
	Returns zero on success, an error code otherwise.
\end{apiret}

\begin{apidep}
	\item osenv_mem_alloc		\S~\ref{dev}
	\item osenv_mem_free		\S~\ref{dev}
	\item osenv_log			\S~\ref{dev}
	\item osenv_assert		\S~\ref{dev}
	\item memcpy			\S~\ref{memcpy}
	\item memset			\S~\ref{memset}
\end{apidep}

