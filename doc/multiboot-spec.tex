\begin{verbatim}
                  Excerpt from "MultiBoot Standard"
                             Version 0.6

                            March 29, 1996
     _________________________________________________________________
                                      
(This contains the essential MultiBoot specification, omitting background
and related info found in ftp://flux.cs.utah.edu/flux/multiboot/.)

Contents
     * Terminology
     * Scope and Requirements
     * Details
     * Authors

     The following items are not part of the standards document,
     but are included for prospective OS and bootloader writers.
     * Example OS Code
     * Example Bootloader Code
       
     _________________________________________________________________
                                      
Terminology

   Throughout this document, the term "boot loader" means whatever
   program or set of programs loads the image of the final operating
   system to be run on the machine. The boot loader may itself consist of
   several stages, but that is an implementation detail not relevant to
   this standard. Only the "final" stage of the boot loader - the stage
   that eventually transfers control to the OS - needs to follow the
   rules specified in this document in order to be "MultiBoot compliant";
   earlier boot loader stages can be designed in whatever way is most
   convenient.
   
   The term "OS image" is used to refer to the initial binary image that
   the boot loader loads into memory and transfers control to to start
   the OS. The OS image is typically an executable containing the OS
   kernel.
   
   The term "boot module" refers to other auxiliary files that the boot
   loader loads into memory along with the OS image, but does not
   interpret in any way other than passing their locations to the OS when
   it is invoked.
   
     _________________________________________________________________
                                      
Scope and Requirements

  Architectures
  
   This standard is primarily targetted at PC's, since they are the most
   common and have the largest variety of OS's and boot loaders. However,
   to the extent that certain other architectures may need a boot
   standard and do not have one already, a variation of this standard,
   stripped of the x86-specific details, could be adopted for them as
   well.
   
  Operating systems
  
   This standard is targetted toward free 32-bit operating systems that
   can be fairly easily modified to support the standard without going
   through lots of bureaucratic rigmarole. The particular free OS's that
   this standard is being primarily designed for are Linux, FreeBSD,
   NetBSD, Mach, and VSTa. It is hoped that other emerging free OS's will
   adopt it from the start, and thus immediately be able to take
   advantage of existing boot loaders. It would be nice if commercial
   operating system vendors eventually adopted this standard as well, but
   that's probably a pipe dream.
   
  Boot sources
  
   It should be possible to write compliant boot loaders that load the OS
   image from a variety of sources, including floppy disk, hard disk, and
   across a network.
   
   Disk-based boot loaders may use a variety of techniques to find the
   relevant OS image and boot module data on disk, such as by
   interpretation of specific file systems (e.g. the BSD/Mach boot
   loader), using precalculated "block lists" (e.g. LILO), loading from a
   special "boot partition" (e.g. OS/2), or even loading from within
   another operating system (e.g. the VSTa boot code, which loads from
   DOS). Similarly, network-based boot loaders could use a variety of
   network hardware and protocols.
   
   It is hoped that boot loaders will be created that support multiple
   loading mechanisms, increasing their portability, robustness, and
   user-friendliness.
   
  Boot-time configuration
  
   It is often necessary for one reason or another for the user to be
   able to provide some configuration information to the OS dynamically
   at boot time. While this standard should not dictate how this
   configuration information is obtained by the boot loader, it should
   provide a standard means for the boot loader to pass such information
   to the OS.
   
  Convenience to the OS
  
   OS images should be easy to generate. Ideally, an OS image should
   simply be an ordinary 32-bit executable file in whatever file format
   the OS normally uses. It should be possible to 'nm' or disassemble OS
   images just like normal executables. Specialized tools should not be
   needed to create OS images in a "special" file format. If this means
   shifting some work from the OS to the boot loader, that is probably
   appropriate, because all the memory consumed by the boot loader will
   typically be made available again after the boot process is created,
   whereas every bit of code in the OS image typically has to remain in
   memory forever. The OS should not have to worry about getting into
   32-bit mode initially, because mode switching code generally needs to
   be in the boot loader anyway in order to load OS data above the 1MB
   boundary, and forcing the OS to do this makes creation of OS images
   much more difficult.
   
   Unfortunately, there is a horrendous variety of executable file
   formats even among free Unix-like PC-based OS's - generally a
   different format for each OS. Most of the relevant free OS's use some
   variant of a.out format, but some are moving to ELF. It is highly
   desirable for boot loaders not to have to be able to interpret all the
   different types of executable file formats in existence in order to
   load the OS image - otherwise the boot loader effectively becomes
   OS-specific again.
   
   This standard adopts a compromise solution to this problem. MultiBoot
   compliant boot images always either (a) are in ELF format, or (b)
   contain a "magic MultiBoot header", described below, which allows the
   boot loader to load the image without having to understand numerous
   a.out variants or other executable formats. This magic header does not
   need to be at the very beginning of the executable file, so kernel
   images can still conform to the local a.out format variant in addition
   to being MultiBoot compliant.
   
  Boot modules
  
   Many modern operating system kernels, such as those of VSTa and Mach,
   do not by themselves contain enough mechanism to get the system fully
   operational: they require the presence of additional software modules
   at boot time in order to access devices, mount file systems, etc.
   While these additional modules could be embedded in the main OS image
   along with the kernel itself, and the resulting image be split apart
   manually by the OS when it receives control, it is often more
   flexible, more space-efficient, and more convenient to the OS and user
   if the boot loader can load these additional modules independently in
   the first place.
   
   Thus, this standard should provide a standard method for a boot loader
   to indicate to the OS what auxiliary boot modules were loaded, and
   where they can be found. Boot loaders don't have to support multiple
   boot modules, but they are strongly encouraged to, because some OS's
   will be unable to boot without them.
   
     _________________________________________________________________
                                      
Details

There are three main aspects of the boot-loader/OS image interface this
standard must specify:

     * The format of the OS image as seen by the boot loader.
     * The state of the machine when the boot loader starts the OS.
     * The format of the information passed by the boot loader to the OS.
       
  OS Image Format
  
An OS image is generally just an ordinary 32-bit executable file in the
standard format for that particular OS, except that it may be linked at a
non-default load address to avoid loading on top of the PC's I/O region or
other reserved areas, and of course it can't use shared libraries or other
fancy features. Initially, only images in a.out format are supported; ELF
support will probably later be specified in the standard.

Unfortunately, the exact meaning of the text, data, bss, and entry fields of
a.out headers tends to vary widely between different executable flavors, and it
is sometimes very difficult to distinguish one flavor from another (e.g. Linux
ZMAGIC executables and Mach ZMAGIC executables). Furthermore, there is no
simple, reliable way of determining at what address in memory the text segment
is supposed to start. Therefore, this standard requires that an additional
header, known as a 'multiboot_header', appear somewhere near the beginning of
the executable file. In general it should come "as early as possible", and is
typically embedded in the beginning of the text segment after the "real"
executable header. It _must_ be contained completely within the first 8192
bytes of the executable file, and must be longword (32-bit) aligned. These
rules allow the boot loader to find and synchronize with the text segment in
the a.out file without knowing beforehand the details of the a.out variant. The
layout of the header is as follows:

        +-------------------+
0       | magic: 0x1BADB002 |   (required)
4       | flags             |   (required)
8       | checksum          |   (required)
        +-------------------+
8       | header_addr       |   (present if flags[16] is set)
12      | load_addr         |   (present if flags[16] is set)
16      | load_end_addr     |   (present if flags[16] is set)
20      | bss_end_addr      |   (present if flags[16] is set)
24      | entry_addr        |   (present if flags[16] is set)
        +-------------------+

All fields are in little-endian byte order, of course. The first field is the
magic number identifying the header, which must be the hex value 0x1BADB002.

The flags field specifies features that the OS image requests or requires of
the boot loader. Bits 0-15 indicate requirements; if the boot loader sees any
of these bits set but doesn't understand the flag or can't fulfill the
requirements it indicates for some reason, it must notify the user and fail to
load the OS image. Bits 16-31 indicate optional features; if any bits in this
range are set but the boot loader doesn't understand them, it can simply ignore
them and proceed as usual. Naturally, all as-yet-undefined bits in the flags
word must be set to zero in OS images. This way, the flags fields serves for
version control as well as simple feature selection.

If bit 0 in the flags word is set, then all boot modules loaded along with the
OS must be aligned on page (4KB) boundaries. Some OS's expect to be able to map
the pages containing boot modules directly into a paged address space during
startup, and thus need the boot modules to be page-aligned.

If bit 1 in the flags word is set, then information on available memory via at
least the 'mem_*' fields of the multiboot_info structure defined below must be
included. If the bootloader is capable of passing a memory map (the 'mmap_*'
fields) and one exists, then it must be included as well.

If bit 16 in the flags word is set, then the fields at offsets 8-24 in the
multiboot_header are valid, and the boot loader should use them instead of the
fields in the actual executable header to calculate where to load the OS image.
This information does not need to be provided if the kernel image is in ELF
format, but it should be provided if the images is in a.out format or in some
other format. Compliant boot loaders must be able to load images that either
are in ELF format or contain the load address information embedded in the
multiboot_header; they may also directly support other executable formats, such
as particular a.out variants, but are not required to.

All of the address fields enabled by flag bit 16 are physical addresses. The
meaning of each is as follows:

     * header_addr -- Contains the address corresponding to the beginning
       of the multiboot_header - the physical memory location at which
       the magic value is supposed to be loaded. This field serves to
       "synchronize" the mapping between OS image offsets and physical
       memory addresses.
     * load_addr -- Contains the physical address of the beginning of the
       text segment. The offset in the OS image file at which to start
       loading is defined by the offset at which the header was found,
       minus (header_addr - load_addr). load_addr must be less than or
       equal to header_addr.
     * load_end_addr -- Contains the physical address of the end of the
       data segment. (load_end_addr - load_addr) specifies how much data
       to load. This implies that the text and data segments must be
       consecutive in the OS image; this is true for existing a.out
       executable formats.
     * bss_end_addr -- Contains the physical address of the end of the
       bss segment. The boot loader initializes this area to zero, and
       reserves the memory it occupies to avoid placing boot modules and
       other data relevant to the OS in that area.
     * entry -- The physical address to which the boot loader should jump
       in order to start running the OS.
       
The checksum is a 32-bit unsigned value which, when added to the other required
fields, must have a 32-bit unsigned sum of zero.

  Machine State
  
When the boot loader invokes the 32-bit operating system, the machine must have
the following state:

     * CS must be a 32-bit read/execute code segment with an offset of 0
       and a limit of 0xffffffff.
     * DS, ES, FS, GS, and SS must be a 32-bit read/write data segment
       with an offset of 0 and a limit of 0xffffffff.
     * The address 20 line must be usable for standard linear 32-bit
       addressing of memory (in standard PC hardware, it is wired to 0 at
       bootup, forcing addresses in the 1-2 MB range to be mapped to the
       0-1 MB range, 3-4 is mapped to 2-3, etc.).
     * Paging must be turned off.
     * The processor interrupt flag must be turned off.
     * EAX must contain the magic value 0x2BADB002; the presence of this
       value indicates to the OS that it was loaded by a
       MultiBoot-compliant boot loader (e.g. as opposed to another type
       of boot loader that the OS can also be loaded from).
     * EBX must contain the 32-bit physical address of the multiboot_info
       structure provided by the boot loader (see below).
       
All other processor registers and flag bits are undefined. This includes, in
particular:

     * ESP: the 32-bit OS must create its own stack as soon as it needs
       one.
     * GDTR: Even though the segment registers are set up as described
       above, the GDTR may be invalid, so the OS must not load any
       segment registers (even just reloading the same values!) until it
       sets up its own GDT.
     * IDTR: The OS must leave interrupts disabled until it sets up its
       own IDT.
       
However, other machine state should be left by the boot loader in "normal
working order", i.e. as initialized by the BIOS (or DOS, if that's what the
boot loader runs from). In other words, the OS should be able to make BIOS
calls and such after being loaded, as long as it does not overwrite the BIOS
data structures before doing so. Also, the boot loader must leave the PIC
programmed with the normal BIOS/DOS values, even if it changed them during the
switch to 32-bit mode.

  Boot Information Format
  
Upon entry to the OS, the EBX register contains the physical address of a
'multiboot_info' data structure, through which the boot loader communicates
vital information to the OS. The OS can use or ignore any parts of the
structure as it chooses; all information passed by the boot loader is advisory
only.

The multiboot_info structure and its related substructures may be placed
anywhere in memory by the boot loader (with the exception of the memory
reserved for the kernel and boot modules, of course). It is the OS's
responsibility to avoid overwriting this memory until it is done using it.

The format of the multiboot_info structure (as defined so far) follows:

        +-------------------+
0       | flags             |   (required)
        +-------------------+
4       | mem_lower         |   (present if flags[0] is set)
8       | mem_upper         |   (present if flags[0] is set)
        +-------------------+
12      | boot_device       |   (present if flags[1] is set)
        +-------------------+
16      | cmdline           |   (present if flags[2] is set)
        +-------------------+
20      | mods_count        |   (present if flags[3] is set)
24      | mods_addr         |   (present if flags[3] is set)
        +-------------------+
28 - 40 | syms              |   (present if flags[4] or flags[5] is set)
        +-------------------+
44      | mmap_length       |   (present if flags[6] is set)
48      | mmap_addr         |   (present if flags[6] is set)
        +-------------------+

The first longword indicates the presence and validity of other fields in the
multiboot_info structure. All as-yet-undefined bits must be set to zero by the
boot loader. Any set bits that the OS does not understand should be ignored.
Thus, the flags field also functions as a version indicator, allowing the
multiboot_info structure to be expanded in the future without breaking
anything.

If bit 0 in the multiboot_info.flags word is set, then the 'mem_*' fields are
valid. 'mem_lower' and 'mem_upper' indicate the amount of lower and upper
memory, respectively, in kilobytes. Lower memory starts at address 0, and upper
memory starts at address 1 megabyte. The maximum possible value for lower
memory is 640 kilobytes. The value returned for upper memory is maximally the
address of the first upper memory hole minus 1 megabyte. It is not guaranteed
to be this value.

If bit 1 in the multiboot_info.flags word is set, then the 'boot_device' field
is valid, and indicates which BIOS disk device the boot loader loaded the OS
from. If the OS was not loaded from a BIOS disk, then this field must not be
present (bit 3 must be clear). The OS may use this field as a hint for
determining its own "root" device, but is not required to. The boot_device
field is layed out in four one-byte subfields as follows:

        +-------+-------+-------+-------+
        | drive | part1 | part2 | part3 |
        +-------+-------+-------+-------+

The first byte contains the BIOS drive number as understood by the BIOS INT
0x13 low-level disk interface: e.g. 0x00 for the first floppy disk or 0x80 for
the first hard disk.

The three remaining bytes specify the boot partition. 'part1' specifies the
"top-level" partition number, 'part2' specifies a "sub-partition" in the
top-level partition, etc. Partition numbers always start from zero. Unused
partition bytes must be set to 0xFF. For example, if the disk is partitioned
using a simple one-level DOS partitioning scheme, then 'part1' contains the DOS
partition number, and 'part2' and 'part3' are both zero. As another example, if
a disk is partitioned first into DOS partitions, and then one of those DOS
partitions is subdivided into several BSD partitions using BSD's "disklabel"
strategy, then 'part1' contains the DOS partition number, 'part2' contains the
BSD sub-partition within that DOS partition, and 'part3' is 0xFF.

DOS extended partitions are indicated as partition numbers starting from 4 and
increasing, rather than as nested sub-partitions, even though the underlying
disk layout of extended partitions is hierarchical in nature. For example, if
the boot loader boots from the second extended partition on a disk partitioned
in conventional DOS style, then 'part1' will be 5, and 'part2' and 'part3' will
both be 0xFF.

If bit 2 of the flags longword is set, the 'cmdline' field is valid, and
contains the physical address of the the command line to be passed to the
kernel. The command line is a normal C-style null-terminated string.

If bit 3 of the flags is set, then the 'mods' fields indicate to the kernel
what boot modules were loaded along with the kernel image, and where they can
be found. 'mods_count' contains the number of modules loaded; 'mods_addr'
contains the physical address of the first module structure. 'mods_count' may
be zero, indicating no boot modules were loaded, even if bit 1 of 'flags' is
set. Each module structure is formatted as follows:

        +-------------------+
0       | mod_start         |
4       | mod_end           |
        +-------------------+
8       | string            |
        +-------------------+
12      | reserved (0)      |
        +-------------------+

The first two fields contain the start and end addresses of the boot module
itself. The 'string' field provides an arbitrary string to be associated with
that particular boot module; it is a null-terminated ASCII string, just like
the kernel command line. The 'string' field may be 0 if there is no string
associated with the module. Typically the string might be a command line (e.g.
if the OS treats boot modules as executable programs), or a pathname (e.g. if
the OS treats boot modules as files in a file system), but its exact use is
specific to the OS. The 'reserved' field must be set to 0 by the boot loader
and ignored by the OS.

NOTE: Bits 4 & 5 are mutually exclusive.

If bit 4 in the multiboot_info.flags word is set, then the following fields in
the multiboot_info structure starting at byte 28 are valid:

        +-------------------+
28      | tabsize           |
32      | strsize           |
36      | addr              |
40      | reserved (0)      |
        +-------------------+

These indicate where the symbol table from an a.out kernel image can be found.
'addr' is the physical address of the size (4-byte unsigned long) of an array
of a.out-format 'nlist' structures, followed immediately by the array itself,
then the size (4-byte unsigned long) of a set of null-terminated ASCII strings
(plus sizeof(unsigned long) in this case), and finally the set of strings
itself. 'tabsize' is equal to it's size parameter (found at the beginning of
the symbol section), and 'strsize' is equal to it's size parameter (found at
the beginning of the string section) of the following string table to which the
symbol table refers. Note that 'tabsize' may be 0, indicating no symbols, even
if bit 4 in the flags word is set.

If bit 5 in the multiboot_info.flags word is set, then the following fields in
the multiboot_info structure starting at byte 28 are valid:

        +-------------------+
28      | num               |
32      | size              |
36      | addr              |
40      | shndx             |
        +-------------------+

These indicate where the section header table from an ELF kernel is, the size
of each entry, number of entries, and the string table used as the index of
names. They correspond to the 'shdr_*' entries ('shdr_num', etc.) in the
Executable and Linkable Format (ELF) specification in the program header. All
sections are loaded, and the physical address fields of the elf section header
then refer to where the sections are in memory (refer to the i386 ELF
documentation for details as to how to read the section header(s)). Note that
'shdr_num' may be 0, indicating no symbols, even if bit 5 in the flags word is
set.

If bit 6 in the multiboot_info.flags word is set, then the 'mmap_*' fields are
valid, and indicate the address and length of a buffer containing a memory map
of the machine provided by the BIOS. 'mmap_addr' is the address, and
'mmap_length' is the total size of the buffer. The buffer consists of one or
more of the following size/structure pairs ('size' is really used for skipping
to the next pair):

        +-------------------+
-4      | size              |
        +-------------------+
0       | BaseAddrLow       |
4       | BaseAddrHigh      |
8       | LengthLow         |
12      | LengthHigh        |
16      | Type              |
        +-------------------+

where 'size' is the size of the associated structure in bytes, which can be
greater than the minimum of 20 bytes. 'BaseAddrLow' is the lower 32 bits of the
starting address, and 'BaseAddrHigh' is the upper 32 bits, for a total of a
64-bit starting address. 'LengthLow' is the lower 32 bits of the size of the
memory region in bytes, and 'LengthHigh' is the upper 32 bits, for a total of a
64-bit length. 'Type' is the variety of address range represented, where a
value of 1 indicates available RAM, and all other values currently indicated a
reserved area.

The map provided is guaranteed to list all standard RAM that should be
available for normal use.

  __________________________________________________________________________
                                       
Authors

Bryan Ford
Flux Research Group
Dept. of Computer Science
University of Utah
Salt Lake City, UT 84112
multiboot@flux.cs.utah.edu
baford@cs.utah.edu

Erich Stefan Boleyn
924 S.W. 16th Ave, #202
Portland, OR, USA  97205
(503) 226-0741
erich@uruk.org

We would also like to thank the many other people have provided comments,
ideas, information, and other forms of support for our work.

  __________________________________________________________________________
                                       
Example OS code can be found in the OSKit in the "kern/x86" directory and
in the oskit/x86/multiboot.h file.

  __________________________________________________________________________
                                       
Example Bootloader Code (from Erich Boleyn) - The GRUB bootloader
project (http://www.uruk.org/grub) will be fully Multiboot-compliant,
supporting all required and optional features present in this
standard.  A final release has not been made, but the GRUB beta release
(which is quite stable) is available from ftp://ftp.uruk.org/public/grub/.
\end{verbatim}
