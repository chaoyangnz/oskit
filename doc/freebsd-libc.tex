%
% Copyright (c) 1997-2000 University of Utah and the Flux Group.
% All rights reserved.
% 
% The University of Utah grants you the right to copy and reproduce this
% document or portions thereof for academic, research, evaluation, and
% personal use only, provided that (1) the title page appears prominently,
% and (2) these copyright and permission notices are retained in all copies.
% To arrange for alternate terms, contact the University of Utah at
% csl-dist@cs.utah.edu or +1-801-585-3271.
%
\label{freebsd-libc}

\section{Introduction}

The \freebsd{} C library is provided as an alternative to the \oskit's
minimal C library (see Section~\ref{libc}) so that more sophisicated
applications can be built. It is derived from version 2.2.2. In addition to
the standard single threaded version of the library, a multi threaded
version is also built which relies on the pthread library (see
Section~\ref{pthread}) to supply the locking primitives.
Both of these libraries
can be found in the lib directory as oskit_freebsd_c.a and
oskit_freebsd_c_r.a. In order to link with the \freebsd{} C library, the
application must be compiled against the \freebsd{} C header files. Example
kernels that are built with the \freebsd{} libraries can be found in the
examples/extended and examples/threads directories.

The following sections briefly describe the \oskit's implementation of the
\freebsd{} C library. Not all of the library is built since some parts do not
make sense in the \oskit's basic environment. Those functions are listed
below, as well as a list of the extended initialization functions.

\section{\textnormal{\posix{}} Interface}

Like the minimal C library, the \freebsd{} C library depends on the
\posix{} library (see Section~\ref{posix-lib}) to provide mappings to the
appropriate \oskit{} COM interfaces. For example, \texttt{fopen} in the C
library will chain to \texttt{open} in the \posix{} library, which in turn
will chain to the appropriate {\tt oskit_dir} and {\tt oskit_file} COM
operations. Applications that link with the \freebsd{} C library must also
link with the COM library (but not the \posix{} library since that is
included as part of the \freebsd{} C library archive file). A
multi-threaded version of the \posix{} library is also provided for
applications that link with the multi-threaded version of the \freebsd{} C
library.

\section{Malloc Support}

The \freebsd{} malloc has been completely replaced with the \oskit's basic
memory allocator. Please refer to to Section~\ref{memalloc} for a
description of the \oskit's allocator interface. \emph{This is a temporary
measure; a future release will include a more traditional ``fast'' memory
allocator.}

\section{Signal Support}
\label{freebsd-signals}

Rudimentary signal support is provided in both the single and
multi-threaded versions of the library. As part of the C library
initialization, a delivery handler is provided to the kernel library that
is used to pass up hardware exceptions (see Section~\ref{kern-signals}).
Assuming the application has made the necessary calls to {\tt sigaction}
(see Section~\ref{posix-signals} in the \posix{} library documentation) to
arrange for catching signals, an exception causes the delivery function to
be invoked, which converts the machine trap state into a more standard {\tt
sigcontext} structure, and passes that to the application via the signal
handler. The application can freely modify the sigcontext structure; the
sigcontext is copied back into the trap state when the handler returns,
which then becomes the new machine state. Use caution! Note that the
default action for \emph{all} signals is to call panic and reboot the
machine. Any hardware exception that that results in a signal that is
blocked, also generates a panic and reboot.

\section{Missing Functionality}

Not all of the \freebsd{} C library has been compiled. In some cases, the
functions missing simply cannot be implemented in the \oskit's basic
environment. In other cases, they are on the yet to be done list, and will
eventually be added. The list of the missing functions follows is:

All of the external data representation (xdr) functions, all of the remote
procedure call (rpc) functions, gethostid, sethostid, getwd, killpg,
setpgrp, setrgid, setruid, sigvec, sigpause, catopen, catclose, catgets,
clock, confstr, crypt, ctermid, daemon, devname, errlst, execve, execl,
execlp, execle, exect, execv, execvp, getfsent, getfsspec, getfsfile,
setfsent, endfsent, getbootfile, getbsize, cgetent, cgetset, cgetmatch,
cgetcap, cgetnum, cgetstr, cgetustr, cgetfirst, cgetnext, cgetclose getcwd,
getdomainname, getgrent, getgrnam, getgrgid, setgroupent, setgrent,
endgrent, getlogin, getmntinfo, getnetgrent, innetgr, setnetgrent,
endnetgrent getosreldate, getpass, getpwent, getpwnam, getpwuid,
setpassent, setpwent, endpwent, getttyent, getttynam, setttyent, endttyent,
getusershell, setusershell, endusershell, getvfsbyname, getvfsbytype,
getvfsent, setvfsent, endvfsent, vfsisloadable, vfsload, glob, globfree,
initgroups, msgctl, msgget, msgrcv, msgsnd, nice, nlist, ntp_gettime,
pause, popen, psignal, user_from_uid, group_from_gid, scandir, seekdir,
semconfig, semctl, semget, semop, setdomainname, sethostname, longjmperror,
setmode, getmode, shmat, shmctl, shmdt, shmget, siginterrupt, siglist,
sleep, sysconf, sysctl, times, timezone, ttyname, ttyslot, ualarm, unvis,
usleep, valloc, vis, wait, wait3, and waitpid.

\section{errno.h}
% Stolen from Minimal C library documentation.

The symbolic constants defined in {\tt errno.h} have been redefined with
the corresponding symbols defined in \texttt{oskit/error.h}
(see~\ref{oskit-error-h}), which are the error codes used through the \oskit's
COM interfaces; this way, error codes from arbitrary \oskit{} components
can be used directly as \texttt{errno} values at least by programs that use
the FreeBSD C library. The main disadvantage of using COM error codes as
\texttt{errno} values is that, since they don't start from around 0 like
typical Unix errno values, it's impossible to provide a traditional
Unix-style \texttt{sys_errlist} table for them.  However, they are fully
compatible with the \texttt{strerror} and \texttt{perror} routines.

\apisec{Client Operating System Dependencies}

The \freebsd{} C library (and by extension the \posix{} library) derive a
number of its external interfaces from the client OS (see
Section~\ref{clientos}).
Rather than relying on linktime dependencies for these
interfaces, the C library uses a services database (see
Section~\ref{oskit-services}) that was provided when it was initialized. This
services database is pre-loaded (by the client OS when the kernel boots)
with certain COM objects that make the C library functional. One such
interface object is the \texttt{oskit_libcenv_t} COM interface, which
provides hooks that are very specific to the \posix{} and C libraries. This,
and other interfaces, are looked up in the database as required.  For
example, the malloc implementation depends on a lower level ``memory
object'' to satisfy memory requests. The first call to \texttt{malloc} will
result in a lookup in the services database for the \texttt{oskit_mem_t}
COM object. The malloc routines then use this lower level allocator
interface. The intent of this ``indirection'' is to reduce (eventually to
zero) the number of linktime dependencies between the C/Posix library and
the rest of the \oskit{} libraries.

\apisec{Library Initialization}

Typically, the C library is initialized by the Client OS library when the
kernel boots (See Section~\ref{clientos}.) The initialization function is
passed a reference to an \texttt{oskit_services_t} COM object (see
Section~\ref{oskit-services}),
which is a services database that has been preloaded
with a number of interface objects that the C library needs to operate
properly. 

%%%%%%%%%%%%%%%%%%%%%%%%%%%%%%%%%%%%%%%%%%%%%%%%%%%%%%%%%%%%%%%%%%%%%%%%%%%%
%	oskit_load_libc
%
\api{oskit_init_libc}{Load the \freebsd{} C library}
\label{oskit-init-freebsd-libc}
\begin{apisyn}
        \cinclude{oskit/c/fs.h}

        \funcproto void oskit_load_libc(oskit_services_t *services);
\end{apisyn}
\begin{apidesc}
	\texttt{oskit_load_libc} allows for internal initializatons to be
	done. This routine \emph{must} be called when the operating system is
	initialized, typically from the Client OS library. The
	\texttt{services} database is used to lookup other interfaces
	required by the C library, and is maintained as internal state to
	the library. 
\end{apidesc}
\begin{apiparm}
	\item[services] A reference to a services database preloaded with
	interfaces required by the C library.
\end{apiparm}

%%%%%%%%%%%%%%%%%%%%%%%%%%%%%%%%%%%%%%%%%%%%%%%%%%%%%%%%%%%%%%%%%%%%%%%%%%%%
%	oskit_init_libc
%
\api{oskit_init_libc}{Secondary Initialization of the \freebsd{} C library}
\begin{apisyn}
        \funcproto void oskit_init_libc(void);
\end{apisyn}
\begin{apidesc}
	\texttt{oskit_init_libc} allows for secondary initializations to be
	performed by the C library, in cases where lazy initialization is
	not appropriate. It must be called sometime after
	\texttt{oskit_load_libc}.
\end{apidesc}
